%%=============================================================================
%% Samenvatting
%%=============================================================================

% TODO: De "abstract" of samenvatting is een kernachtige (~ 1 blz. voor een
% thesis) synthese van het document.
%
% Een goede abstract biedt een kernachtig antwoord op volgende vragen:
%
% 1. Waarover gaat de bachelorproef?
% 2. Waarom heb je er over geschreven?
% 3. Hoe heb je het onderzoek uitgevoerd?
% 4. Wat waren de resultaten? Wat blijkt uit je onderzoek?
% 5. Wat betekenen je resultaten? Wat is de relevantie voor het werkveld?
%
% Daarom bestaat een abstract uit volgende componenten:
%
% - inleiding + kaderen thema
% - probleemstelling
% - (centrale) onderzoeksvraag
% - onderzoeksdoelstelling
% - methodologie
% - resultaten (beperk tot de belangrijkste, relevant voor de onderzoeksvraag)
% - conclusies, aanbevelingen, beperkingen
%
% LET OP! Een samenvatting is GEEN voorwoord!

%%---------- Samenvatting -----------------------------------------------------

\chapter*{\IfLanguageName{dutch}{Samenvatting}{Abstract}}
In de huidige, sterk gedigitaliseerde wereld, waarin applicaties continu met elkaar communiceren en data uitwisselen, spelen Application Programming Interfaces (API's) een cruciale rol. Een goed ontworpen en gedocumenteerde API is essentieel voor efficiënte integraties, vlotte datastromen en een schaalbare softwarearchitectuur. Dit is in het bijzonder relevant voor BrightAnalytics, een bedrijf dat een geavanceerd data-visualisatieplatform aanbiedt voor financiële rapportage en business intelligence. De kwaliteit, consistentie en efficiëntie van hun API's zijn van cruciaal belang voor de functionaliteit van hun producten, de integratie met andere systemen en de snelle ontwikkeling van nieuwe features.

\bigskip

Deze bachelorproef onderzoekt de praktische voordelen van RESTful API design, met name HATEOAS (Hypermedia as the Engine of Application State) en OpenAPI specificatie, voor de kwaliteit, consistentie en onderhoudbaarheid van een API in Laravel, de backend-technologie die BrightAnalytics gebruikt. Welke concrete voordelen bieden deze principes voor de ontwikkeling en het onderhoud van een API? Hoe beïnvloeden ze de efficiëntie en de kost van ontwikkeling? Welke
impact hebben ze op de frontend-ontwikkeling en de gebruikerservaring? Zijn er principes die een te grote overhead met zich meebrengen, of die niet relevant zijn in de tech stack Laravel/Vue.js?

\bigskip

De centrale onderzoeksvraag luidt: "Welke concrete voordelen biedt de implementatie van RESTful API design, en in het bijzonder HATEOAS en OpenAPI, voor de kwaliteit, consistentie en onderhoudbaarheid van een Laravel-API en een bijbehorende frontend applicatie?"

\bigskip

Om deze vraag te beantwoorden, werden de volgende onderzoeksdoelstellingen geformuleerd:

\begin{itemize}
  \item een literatuurstudie uitvoeren naar RESTful API design principes, HATEOAS en OpenAPI specificatie;
  \item de BrightEats API ontwikkelen en refactoren volgens deze standaarden
  \item de impact van de implementatie van deze principes op de kwaliteit, consistentie en onderhoudbaarheid van de API evalueren, zowel vanuit backend- als frontend perspectief
  \item de impact op de kost van ontwikkeling en onderhoud van de API evalueren: is de investering in een goed gestructureerde API gerechtvaardigd? Welke RESTful API design principes zijn het meest kostenefficiënt? Zijn er principes die een te grote overhead met zich meebrengen?
  \item op basis van de bevindingen aanbevelingen formuleren voor BrightAnalytics over de toepassing van RESTful API design, HATEOAS en OpenAPI.
\end{itemize}

\bigskip

De methodologie van dit onderzoek omvatte een literatuurstudie naar RESTful API design principes, HATEOAS en OpenAPI, alsook een praktische case study bij BrightAnalytics, waarbij de BrightEats API iteratief werd ontwikkeld en verfijnd. De impact van de implementatie van deze principes op de kwaliteit, consistentie en onderhoudbaarheid van de API werd geëvalueerd, zowel vanuit backend- als frontend perspectief. De bevindingen uit de literatuurstudie werden direct toegepast in de ontwikkeling van de API, en de ervaringen tijdens de ontwikkeling stuurden het onderzoek bij.

\bigskip

% resultaten

\bigskip

% conclusies, aanbevelingen, beperkingen
