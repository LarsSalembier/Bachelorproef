%%=============================================================================
%% Inleiding
%%=============================================================================

\chapter{\IfLanguageName{dutch}{Inleiding}{Introduction}}%
\label{ch:inleiding}

De explosieve groei van data en de toenemende vraag naar naadloze integraties tussen applicaties hebben de rol van Application Programming Interfaces (API's) centraal geplaatst in de hedendaagse softwareontwikkeling. Een goed ontworpen en gedocumenteerde API is niet langer een luxe, maar essentieel voor succes in het digitale landschap. Dit is met name relevant voor BrightAnalytics, een bedrijf dat een geavanceerd data-visualisatieplatform aanbiedt voor financiële rapportering en business intelligence. De kwaliteit, consistentie en efficiëntie van de API's die BrightAnalytics ontwikkelt, zijn van cruciaal belang voor de functionaliteit van hun producten, de integratie met andere systemen en het snel kunnen ontwikkelen van nieuwe features.

\bigskip

Tijdens mijn stage bij BrightAnalytics heb ik "BrightEats" ontwikkeld, een applicatie waarmee werknemers hun lunch kunnen bestellen. Deze applicatie is afhankelijk van een backend API, geschreven in Laravel, net zoals alle andere applicaties bij BrightAnalytics. De ontwikkeling van deze API biedt een uitgelezen kans om de principes van RESTful API design te implementeren en om te evalueren welke voor- en nadelen de meer geavanceerde principes hiervan (met name OpenAPI en Hypermedia As The Engine Of Application State (HATEOAS)) met zich meebrengen. Het is belangrijk om te onderzoeken of de theoretische voordelen van deze principes zich vertalen in praktische winst binnen de specifieke context van BrightAnalytics, rekening houdend met de bestaande technologieën en de ontwikkelprocessen.

\bigskip

Het principe van RESTful API design werd in 2000 geïntroduceerd door Roy Fielding in zijn proefschrift \autocite{Fielding2000}. REST (Representational State Transfer) is ondertussen wijdverspreid en wordt goed ondersteund bij het bouwen van een API met Laravel. Echter, de meer geavanceerde onderdelen van REST, met name HATEOAS (Hypermedia As The Engine Of Application State) en OpenAPI, zijn minder bekend en worden minder vaak toegepast in de praktijk. Deze principes kunnen echter een grote impact hebben op de kwaliteit, consistentie en onderhoudbaarheid van een API. De vraag is of deze impact voldoende positief is om de extra inspanning van implementatie en onderhoud te rechtvaardigen, en of ze compatibel zijn met de bestaande frontend technologie, Vue.js.

\section{\IfLanguageName{dutch}{Probleemstelling}{Problem Statement}}%
\label{sec:probleemstelling}

De ontwikkeling en het onderhoud van kwalitatieve API's is een uitdaging voor veel bedrijven, waaronder BrightAnalytics. Inconsistente API-ontwerpen, gebrekkige documentatie en een suboptimale communicatie tussen frontend- en backend-teams kunnen leiden tot vertragingen in de ontwikkeling en dus verhoogte kosten. Dit onderzoek richt zich specifiek op het ontwikkelteam van BrightAnalytics en onderzoekt hoe RESTful API design principes, HATEOAS en OpenAPI, kunnen bijdragen aan een efficiëntere en kwalitatievere API-ontwikkeling binnen hun Laravel/Vue.js omgeving. De focus ligt op het identificeren van concrete best practices en het evalueren van hun impact op de dagelijkse workflow van de ontwikkelaars.

\section{\IfLanguageName{dutch}{Onderzoeksvraag}{Research question}}%
\label{sec:onderzoeksvraag}

De centrale onderzoeksvraag van deze bachelorproef is: "Welke concrete voordelen biedt de implementatie van RESTful API design, en in het bijzonder HATEOAS en OpenAPI, voor de kwaliteit, consistentie en onderhoudbaarheid van een Laravel-API en een bijbehorende frontend applicatie?"

De volgende deelvragen helpen deze centrale vraag te beantwoorden:
\begin{itemize}
  \item Hoe kunnen de principes van RESTful API design concreet worden toegepast in een Laravel project, en welke tools en libraries zijn hierbij relevant?
  \item Biedt HATEOAS daadwerkelijk een meerwaarde op vlak van flexibiliteit en loose coupling tussen client en server in een Laravel/Vue.js applicatie, en zo ja, wegen deze voordelen op tegen de extra tijd en complexiteit die nodig zijn voor de implementatie?
  \item Hoe kan OpenAPI bijdragen aan een verbeterde documentatie en standaardisatie van de BrightEats API, en hoe kunnen tools hierbij worden ingezet? 
  \item Welke meetbare impact heeft de toepassing van RESTful design, HATEOAS en OpenAPI op de kwaliteit van de code, de snelheid van ontwikkeling en de onderhoudbaarheid van de BrightEats API, bijvoorbeeld op vlak van code complexiteit, aantal bugs en tijd besteed aan debugging?
  \item Welke concrete aanpassingen zijn vereist in de Vue.js frontend om optimaal te kunnen integreren met een HATEOAS-gebaseerde API, en welke invloed hebben deze aanpassingen op de onderhoudbaarheid en robuustheid van de frontend code?
\end{itemize}

\section{\IfLanguageName{dutch}{Onderzoeksdoelstelling}{Research objective}}%
\label{sec:onderzoeksdoelstelling}

Het doel van deze bachelorproef is om praktische inzichten te verwerven in de voor- en nadelen van RESTful API design, HATEOAS en OpenAPI binnen de tech stack Laravel/Vue.js. Door middel van een literatuurstudie en de praktische case study van de BrightEats API, wordt de impact van deze principes op de kwaliteit, consistentie en onderhoudbaarheid van de API geëvalueerd, met een focus op meetbare resultaten. Het uiteindelijke doel is om concrete en haalbare aanbevelingen te formuleren over RESTful API design in Laravel, waarbij we enkel de principes behouden die een duidelijke meerwaarde bieden voor de ontwikkeling en het onderhoud van de API en de frontend applicatie.

\section{\IfLanguageName{dutch}{Opzet van deze bachelorproef}{Structure of this bachelor thesis}}%
\label{sec:opzet-bachelorproef}

% Het is gebruikelijk aan het einde van de inleiding een overzicht te
% geven van de opbouw van de rest van de tekst. Deze sectie bevat al een aanzet
% die je kan aanvullen/aanpassen in functie van je eigen tekst.

De rest van deze bachelorproef is als volgt opgebouwd:

\begin{itemize}
  \item Hoofdstuk~\ref{ch:stand-van-zaken} presenteert een literatuurstudie over REST, HATEOAS en OpenAPI, inclusief best practices en implementatievoorbeelden in Laravel.
  \item Hoofdstuk~\ref{ch:methodologie} beschrijft de methodologie van het onderzoek, inclusief de iteratieve ontwikkelingscyclus van de BrightEats API.
  \item Hoofdstuk~\ref{ch:richtlijnen} formuleert concrete richtlijnen voor het implementeren van RESTful API design, HATEOAS en OpenAPI in een Laravel project, gebaseerd op de literatuurstudie en de praktische ervaringen bij BrightAnalytics.
  \item Hoofdstuk~\ref{ch:implementatie} detailleert de implementatie van RESTful principes en het opstellen van de OpenAPI-specificatie voor de BrightEats API. Hier worden codevoorbeelden, tools en de gemaakte keuzes besproken. Ook de aanpassingen in de frontend applicatie komen aan bod.
  \item Hoofdstuk~\ref{ch:evaluatie} evalueert de impact van de implementatie op kwaliteit, consistentie en onderhoudbaarheid. Hier worden de resultaten van tests, feedback van ontwikkelaars en de kost van implementatie geanalyseerd.
  \item Hoofdstuk~\ref{ch:conclusie} concludeert het onderzoek en formuleert concrete aanbevelingen. Deze aanbevelingen gaan over best practices voor API-ontwikkeling en een afweging van de kosten en baten van RESTful design, HATEOAS en OpenAPI.
\end{itemize}
